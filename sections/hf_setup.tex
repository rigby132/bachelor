\subsection{Hamilton-Operator}
Der allgemeine Hamilton\-/Operator für Möleküle mit $N$ Elektronen und $M$ Atomkernen lautet:
\begin{flalign}\label{hamilton}
  \hat{H} &= -\sum_i^N \frac{1}{2} \nabla_i^2 
            -\sum_\mu^M \frac{1}{2 m_\mu} \nabla_\mu^2
            -\sum_i^N \sum_\mu^M \frac{Z_\mu}{r_{i\mu}}
            +\frac{1}{2}\sum_{i\neq j} \frac{1}{r_{ij}}
            +\frac{1}{2}\sum_{\mu \neq \nu } \frac{Z_\mu Z_\nu}{r_{\mu\nu}}\nonumber\\
          &= \hat{T}_e + \hat{T}_A + \hat{V}_{eA} + \hat{V}_{ee} + \hat{V}_{AA}
\end{flalign}
Dabei steht $Z_\mu$ für die Ladung des Atomkerns $\mu$ und 
$r$ für den Abstand zwischen Elektronen und/oder Atomkernen.
\cite[S. 6]{tc2_1}

TODO: erkläre alle Terme.

\subsection{Born-Oppenheimer-Näherung}
Aufgrund des hohen Masseunterschiedes zwischen Elektronen und Atomkernen
ist der Einfluss der Elektronen auf die Bewegung der trägeren Atomkerne vernachlässigbar.
Deshalb können bei der Berechnung der Elektronen\-/Wellenfunktion 
die Atomkerne approximativ als statisch betrachtet werden.

Dafür wird der Allgemeine Hamilton-Operator \cref{hamilton} aufgeteilt:
\begin{flalign*}
  \hat{H} &= \hat{T}_e + \hat{T}_A + \hat{V}_{eA} + \hat{V}_{ee} + \hat{V}_{AA}\\
          &= \hat{T}_A + \hat{V}_{AA} + \hat{H}_{\text{el}}
\end{flalign*}
Wir lösen nun die Schrödingergleichung mit dem elektronischen Hamilton-Operator $\hat{H}_{\text{el}}$.
Bei der Lösung dieser wird eine feste Kerngeometrie angenommen, 
die wir bei dem Operator $\hat{V}_{eA}$ verwenden werden. 
Mit der resultierenden Elektronen\-/Wellenfunktionen 
lässt sich dann eine Gesamte Wellenfunktion unter Einbezug der Atomkerne konstruieren.
\cite[S. 11-14]{tc2_1}

\subsection{Variationsformulierung}
Da eine analytische Lösung zur Schrödingergleichung nur in speziellen Fällen existiert \cite[S. 195]{lewars_2016},
wird die exakte Wellenfunktion durch eine Test\-/Wellenfunktion approximiert.
Es lässt sich zeigen, 
dass die Energie dieser Test\-/Wellenfunktion $E_{test}$ 
immer über der tatsächlichen Grundzustandsenergie $E_0$ liegt.

\subsubsection*{Beweis}
TODO Auf Rayleigh-ritz erweitern und auf det$|\hat{H}_{ij} - ES_{ij}| = 0$ kommen\\
TODO test funktion normiert annahme -> $E_0$ kann dann locker ins integral.
Voraussetzungen:
\begin{enumerate}
  \item Der Hamiltonian $\hat{H}$ hat die Eigenfunktionen $\psi_i^{}$ mit Eigenwerten $E_i^{}$.
  \item Es existiert eine Eigenfunktion $\psi_0^{}$ mit dem niedrigsten  Eigenwert $E_0^{}$.
  \item Alle Eigenfunktion von $\hat{H}$ sind orthonomal zueinander:\\
  $\langle \psi_i^{} \vert \psi_j^{} \rangle = \delta_{ij}^{},\quad\forall i,j$
  \item Die Test-Wellenfunktion lässt sich als Lineakombination der Eigenfunktionen darstellen:
  $\psi_{test}^{} = \sum_{n}^{} c_n^{} \psi_n^{}$
\end{enumerate}
Zu zeigen: $E_{test}^{} \geq E_0^{}$ oder $E_{test}^{} - E_0^{} \geq 0$
\begin{flalign*}
  E_{test} - E_0 
  &= \langle \psi_{test}^{} \vert \hat{H} - E_0^{} \vert \psi_{test}^{} \rangle\\
  &= \int \psi_{test}^* (\hat{H} - E_0^{}) \psi_{test}^{} \,dx \quad &\vert \text{ 4. Voraussetzung}\\
  &= \sum_n \sum_m c_n^\ast c_m^{} \int \psi_{n}^* (\hat{H} - E_0^{}) \psi_{m} \,dx 
  \quad &\vert \text{ 1. Voraussetzung}\\
  &= \sum_n \sum_m c_n^\ast c_m^{} \int \psi_{n}^* (E_m^{} - E_0^{}) \psi_{m} \,dx 
  \quad &\vert \text{ 3. Voraussetzung}\\
  &= \sum_n \left\lvert c_n^{} \right\rvert^2 (E_n^{} - E_0^{}) \int \psi_{n}^* \psi_{n} \,dx 
  \quad &\vert \text{ 3. Voraussetzung}\\
  &= \sum_n \left\lvert c_n^{} \right\rvert^2 (E_n^{} - E_0^{})
  \quad &\vert \text{ 2. Voraussetzung}\\
  &\geq 0 &\qed
\end{flalign*}
\cite[S. 187]{atkins_friedman_2011}\\
%\cite[]{TC2}
TODO Warum Voraussetzungen erfüllt sind, erklären.

\subsection{Beschreibung von Elektronen}
Die gesamte Wellenfunktion $\Psi$ muss noch einige Eigenschaften erfüllen, 
um die Elektronen des Moleküls zu beschreiben:

\begin{enumerate}
  \item Jedes Elektron verfügt, neben der räumlichen Ausdehnung(TODO: Formulierung), 
  auch über einen intrinsischen Spin. Dieser Spin kann zwei Zustände annehmen, diese werden über
  die Spinfunktionen $\alpha$ und $\beta$ dargestellt.
  \item Die Gesamt\-/Wellenfunktion muss antisymmetrisch sein 
  bezüglich dem Austausch von zwei beliebigen Elektronen z.B.
  $\Psi(1, 2) = - \Psi(2, 1)$
\end{enumerate}
TODO cite\\
Deshalb stellen wir die Gesamt\-/Wellenfunktion mit 
$2n, n \in \mathbb{N}$ Elektronen als eine Determinante von Spin-Orbitalfunktionen dar:
\begin{equation}\label{slater}
\Psi(1, \dots, n) = 
\frac{1}{\sqrt{(2n)!}}\left\lvert
\begin{array}{ccccc} 
\psi_1(1)\alpha(1) & \psi_1(1)\beta(1) & \cdots & \psi_n(1)\alpha(1) & \psi_n(1)\beta(1)\\ 
\psi_1(2)\alpha(2) & \psi_1(2)\beta(2) & \cdots & \psi_n(2)\alpha(2) & \psi_n(2)\beta(2)\\ 
    \vdots         &       \vdots      & \ddots &       \vdots       &       \vdots     \\ 
\psi_1(2n)\alpha(2n) & \psi_1(2n)\beta(2n) & \cdots & \psi_n(2n)\alpha(2n) & \psi_n(2n) \beta(2n)
\end{array}
\right \rvert
\end{equation}

TODO zusatzinfos? Erklärung spin auf s.200 lewars\\
+ eine zeile mit $\varphi_1 = \psi_1\alpha$ und $\varphi_2 = \psi_1\beta$