\section{Allgemeine Theorie}

\subsection{Grundlegende Definitionen und Notation}
Um Moleküle in einem quantenmechanischen Modell zu beschreiben,
müssen wir zunächst einige Definitionen aufstellen:

\subsubsection{Wellenfunktionen}
Die sogenannte Wellenfunktion $\Psi(x_1, x_2, \dots)$ 
beschreibt den Zustand eines quantenmechanischen Systems vollständig.
Im Kontext eines Moleküls entspricht dies der Wellenfunktion für die Atomkerne
und deren Elektronen.

Zu den Wellenfunktionen $\psi, \varphi$ wird ein Skalarprodukt definiert:
\begin{equation}
    \langle \psi(x) \vert \varphi(x) \rangle \equiv  \int \psi^*(x) \phi(x) \,dx
\end{equation}
Das Integral geht über den gesamten Raum aller Parameter.

Das Betragsquadrat der Wellenfunktion $\vert \Psi \vert^2$
lässt sich dabei als Wahrscheinlichkeitsdichte interpretieren, wenn $\Psi$ normiert ist
($\langle \Psi \vert \Psi \rangle = 1$).

\cite[S. 20-21, 24]{atkins_friedman_2011}

In der restlichen Arbeit gehen wir immer von normierten Wellenfunktionen aus.
Sollten die Parameter nicht von direkter Relevanz sein, werden diese weggelassen.

\subsubsection{Operatoren}
Beobachtbare Eigenschaften eines quantenmechanischen Systems 
werden durch sogenannte Operatoren repräsentiert.
Diese Operatoren werden über das Hut\-/Symbol (zb.: $\hat{o}$) gekennzeichnet
und müssen hermitesch sein.
Ein Operator $\hat{o}$ wirkt auf ein quantenmechanisches System $\varphi$
und verändert diesen in einen anderen Zustand $\psi$.

Wir binden hermitesche Operatoren $\hat{o}$ noch in unsere Notation für das Skalarprodukt ein,
sodass sich schreiben lässt:
\begin{equation}
    \langle \psi(x) \vert \hat{o} \vert \varphi(x) \rangle \equiv
    \int \psi^*(x) \hat{o} \varphi(x) \,dx
\end{equation}

Ist das System $\varphi$ eine Eigenfunktion des Operators $\hat{o}$,
erhalten wir eine Eigenwertsgleichung mit dem Eigenwert $\lambda$:
\begin{equation}
    \hat{o} \varphi = \lambda \varphi
\end{equation}
Außerdem gilt für das Skalarprodukt mit sich selbst dann:
\begin{equation}
    \langle \varphi(x) \vert \hat{o} \vert \varphi(x) \rangle = \lambda
\end{equation}

%TODO cite lecture (bookmark)

Sollte $\varphi$ keine Eigenfunktion eines Operators $\hat{o}$ sein,
so kann man den Erwartungswert der Eigenschaft, welche durch $\hat{o}$ ausgedrückt wird, bestimmen über:
\begin{equation}
\langle \psi(x) \vert \hat{o} \vert \psi(x) \rangle
  = \int \psi^*(x) \hat{o} \psi(x) \,dx
\end{equation}

%TODO cite lecture (bookmark)
\cite[S. 21]{atkins_friedman_2011}
\cite[S. 155-156]{levine_2019}

\subsubsection{Dirac-Notation}
Im Rahmen dieser Notation lassen sich Funktionen wie $\varphi$
über sogenannte Bras $\langle \varphi \vert$ und Kets $\vert \varphi \rangle$ schreiben.
Sollte einem Bra $\langle \varphi \vert$ ein Ket $\vert \psi \rangle$ folgen,
so steht dies für das Skalarprodukt $\langle \varphi \vert \psi \rangle$.

Zu beachten ist dabei, dass ein Ket mit dessen enthaltener Funktion äquivalent ist:
\begin{equation}
    \varphi \equiv \vert \varphi \rangle
\end{equation}
Ein Bra hingegen wird als Operation für das Skalarprodukt angesehen.

Die Bra-Ket-Notation wird im Kontext des Skalarprodukts verwendet.
\subsubsection{Die Schrödingergleichung}
Im Rahmen dieser Arbeit gehen wir von nur zeitlich statischen Systemen aus,
welche durch die zeitunabhängige Schrödingergleichung beschrieben werden:
\begin{equation}\label{schroedinger}
  \hat{H}\psi = E\psi
\end{equation}

Durch das Lösen dieser Eigenwerts\-/Gleichung sind wir in der Lage
die Energien und weitere Eigenschaften von Molekülen zu bestimmen.
Der Operator $\hat{H}$ ist der sogenannte Hamiltonian und
gibt die Energie $E$ des Systems wieder.

\cite[S. 24-25]{atkins_friedman_2011}