\section{Allgemeine Theorie}

\subsection{Grundlegende Defininitionen}
\subsection{Postulate der Quantenmechanik}
TODO: $\psi$ als immer normiert annehmen?, Mehr infos + umschreiben?
\subsubsection{Wellenfunktionen}
Die sogenannte Wellenfunktion $\Psi(x_1, x_2, \dots, t)$ 
beschreibt den Zustand eines quantenmechanischen Systems vollständig.\cite[S. 20-21]{atkins_friedman_2011}\\\\
Durch ein $\psi$ wird nur der räumliche Anteil dargestellt.

\subsubsection{Operatoren}
Beobachtbare Eigenschaften eines quantenmechanischen Systems 
werden durch sogenannte Operatoren repräsentiert.
Diese Operatoren müssen hermitesch sein und folgende Gleichungen erfüllen:
\begin{flalign*}
  \hat{q}\hat{p}_{q'} - \hat{p}_{q'}\hat{q} &= i\hbar\delta_{qq'}\\
  \hat{q}\hat{q}' - \hat{q}'\hat{q} &= 0 \\
  \hat{p}_{q}\hat{p}_{q'} - \hat{p}_{q'}\hat{p}_{q} &= 0
\end{flalign*}
Wobei $\hat{q}, \hat{q}' \in \{\hat{x}, \hat{y}, \hat{z}\}$ die Orts-Operatoren stellen
und $\hat{p}_q, \hat{p}_{q'}$ die zugehörigen Impuls-Operatoren sind.\cite[S. 21]{atkins_friedman_2011}

\subsubsection{Messungen}
Der Mittelwert von wiederholten Messungen
entspricht dem Erwartungswert des korrespondierenden Operators auf der Wellenfunktion.\\
Der Erwartungswert eines Operators $\hat{o}$ ist gegeben durch:
\begin{equation}
  \left\langle \hat{o} \right\rangle = \frac{\int \psi^* \hat{o} \psi \,d\tau}{\int \psi^* \psi \,d\tau}
  =\frac{\left\langle \psi^* \vert \hat{o} \vert \psi \right\rangle}{\left\langle \psi^* \vert \psi \right\rangle}
\end{equation}
Wenn $\psi$ eine Eigenfunktion eines Operators $\hat{o}$ ist, erhält man den Eigenwert als Erwartungswert.
\cite[S. 22-23]{atkins_friedman_2011} 

\subsubsection{Bornsche Wahrscheinlichkeitsinterpretation}
Die Wahrscheinlichkeit ein Teilchen in einem Volumenelement $\,d\tau$ zu finden ist geich
$\vert \psi(x) \vert^2$, wenn $\psi$ normiert ist.\cite[S. 24]{atkins_friedman_2011}

\subsubsection{Die Schrödingergleichung}
Die zeitliche Änderung dieser Wellenfunktion $\Psi(x_1, x_2, \dots, t)$ 
wird durch die Schrödingergleichung beschrieben:
\begin{equation}
  i\hbar\frac{\partial\Psi}{\partial t} = \hat{H}\Psi
\end{equation}\\
Sollte die Zeitabhägigkeit der Wellenfunktion trivial sein (TODO erklären?),
kann diese in 2 Funktionen zerlegt werden:
\begin{equation*}
  \Psi(x_1, x_2, \dots, t) = \psi(x_1, x_2, \dots) \exp(\frac{-iEt}{\hbar})
\end{equation*}\\
Dabei ist $\psi$ die Lösung für die zeitunabhängige Schrödingergleichung:
\begin{equation}\label{schroedinger}
  \hat{H}\psi = E\psi
\end{equation}
\cite[S. 24-25]{atkins_friedman_2011}