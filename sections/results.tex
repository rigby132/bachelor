\chapter{Ergebnisse}
Im Folgenden werden einige simple Beispiele betrachtet,
welche mit dem zur Arbeit begleitenden Programm berechnet wurden.
Die Ergebnisse werden anschließend mit Werten aus dem Programm NWChem \cite{nwchem} verglichen,
um die qualitative Korrektheit der Implementierung zu bestätigen.
\section{Aufbau}
Wir schauen uns die folgenden Moleküle an:
\begin{enumerate}
    \item Heliumhydridion (HeH$^+$)
    \item Wasser (H$_2$O)
    \item Ammoniak (NH$_3$)
    \item Methan (CH$_4$)
\end{enumerate}
\begin{figure}[h]
    \centering
\includegraphics[trim=1400 1400 1400 1400, clip, width=0.24\textwidth]{res/HeH/heh_w.png}
\includegraphics[trim=1400 1400 1400 1400, clip, width=0.24\textwidth]{res/H2O/h2o.png}
\includegraphics[trim=1400 1400 1400 1400, clip, width=0.24\textwidth]{res/NH3/nh3_d.png}
\includegraphics[trim=1400 1400 1400 1400, clip, width=0.24\textwidth]{res/CH4/ch4.png}
\caption{Von links nach rechts: HeH$^+$, H$_2$O, NH$_3$, CH$_4$}\label{molecules}
\end{figure}
Diese Moleküle sind verglichen zu
allen möglichen organischen Verbindungen
relativ einfache Konstrukte und dienen zum Testen
der grundlegenden Funktionalität.

Wir berechnen nun für jedes dieser Moleküle die Hartree\-/Fock\-/Energie, 
welche sich aus der elektronischen Energie und
der Internuklearen\-/Abstoßungs\-/Energie zusammen setzt.
Diese Energien vergleichen wir dann mit den
zuverlässigen Ergebnissen aus der NWChem\-/Software.
Es wird lediglich ein 6-311+g** Basis\-/Satz verwendet,
welcher groß genug ist, um einigermaßen genaue Berechnungen zu ermöglichen.
Der Basis\-/Satz entstammt der Basis Set Exchange \cite{bse}.
Bei den Berechnungen wurden für möglichst identische Startbedingungen gesorgt.

\section{Numerische Experimente}
Diese Hartree\-/Fock\-/Energien (in Hartree) wurden berechnet:
\begin{center}
\begin{tabular}{c|c|c|c|c}
          & HeH$^+$ & H$_2$O & NH$_3$ & CH$_4$\\ \hline
    HFcpp & \unit[-2.88095]{Ha} & \unit[-73.88436]{Ha} & \unit[-54.55853]{Ha} & \unit[-39.01310]{Ha} \\
    NWChem & \unit[-2.92904]{Ha} & \unit[-76.04854]{Ha} & \unit[-56.21346]{Ha} & \unit[-40.20852]{Ha} \\ \hline
    Differenz & \unit[0.04809]{Ha} & \unit[2.16418]{Ha} & \unit[1.65493]{Ha} & \unit[1.19542]{Ha}\\
    Abweichung & $1.64\%$ & $2.85\%$ & $2.94\%$ & $2.97\%$
\end{tabular}
\end{center}

Wie klar zu erkennen ist, sind die berechneten Energien der eigenen Implementierung
im Bereich der Ergebnisse der NWChem\-/Software. Es lässt sich schlussfolgern,
dass die Implementierung korrekt ist.

Die deutlich besseren Ergebnisse der NWChem-Software
trotz gleicher Startbedingungen sind sehr wahrscheinlich 
auf Unterschiede in der Implementierung zurückzuführen,
wie zum Beispiel auf den Einsatz zusätzlicher Optimierungen.
Es wäre auch möglich, dass kleinere Fehler in der Implementierung
für diese Abweichungen sorgen.

\section{Visualisierung}
Die Hartree\-/Fock\-/Energien sind nicht die einzigen Ergebnisse,
wir erhalten noch die Orbital\-/Funktionen und deren zugehörigen Energien.
Diese lassen sich in einem Isoflächen\-/Plot gut darstellen.
Die nachstehenden Grafiken wurden mithilfe der Programme Avogadro \cite{avogadro}
und POV-Ray \cite{povray} erzeugt.
\begin{enumerate}\bfseries
\item Heliumhydridion (HeH$^+$):
\begin{figure}[H]
\centering
\includegraphics[trim=300 300 300 300, clip, width=0.45\textwidth]{res/HeH/heh_w0.png}
\includegraphics[trim=300 300 300 300, clip, width=0.45\textwidth]{res/HeH/heh_w1.png}
\caption{Die ersten zwei Orbitale des HeH$^+$\-/Moleküls,
nach aufsteigender Energie sortiert.
\textcolor{blue}{$\blacksquare$} positiv,
\textcolor{red}{$\blacksquare$} negativ.}\label{heh_orbitals}
\end{figure}

\newpage
\item Wasser (H$_2$O):
\begin{figure}[H]
\centering
\includegraphics[trim=700 800 700 600, clip, width=0.45\textwidth]{res/H2O/h2o_w0.png}
\includegraphics[trim=700 800 700 600, clip, width=0.45\textwidth]{res/H2O/h2o_w1.png}\\
\includegraphics[trim=700 800 700 600, clip, width=0.45\textwidth]{res/H2O/h2o_w2.png}
\includegraphics[trim=700 800 700 600, clip, width=0.45\textwidth]{res/H2O/h2o_w3.png}\\
\includegraphics[trim=700 800 700 600, clip, width=0.45\textwidth]{res/H2O/h2o_w4.png}
\caption{Die fünf besetzten Orbitale des H$_2$O\-/Moleküls,
nach aufsteigender Energie sortiert.
\textcolor{blue}{$\blacksquare$} positiv,
\textcolor{red}{$\blacksquare$} negativ.}\label{h2o_orbitals}
\end{figure}

\newpage
\item Ammoniak (NH$_3$):
\begin{figure}[H]
\centering
\includegraphics[trim=700 1200 1200 700, clip, width=0.45\textwidth]{res/NH3/nh3_w0.png}
\includegraphics[trim=700 1200 1200 700, clip, width=0.45\textwidth]{res/NH3/nh3_w1.png}\\
\includegraphics[trim=700 1200 1200 700, clip, width=0.45\textwidth]{res/NH3/nh3_w2.png}
\includegraphics[trim=700 1200 1200 700, clip, width=0.45\textwidth]{res/NH3/nh3_w3.png}\\
\includegraphics[trim=700 1200 1200 700, clip, width=0.45\textwidth]{res/NH3/nh3_w4.png}
\caption{Die fünf besetzten Orbitale des NH$_3$\-/Moleküls,
nach aufsteigender Energie sortiert.
\textcolor{blue}{$\blacksquare$} positiv,
\textcolor{red}{$\blacksquare$} negativ.}\label{nh3_orbitals}
\end{figure}

\newpage
\item Methan (CH$_4$):
\begin{figure}[H]
\centering
\includegraphics[trim=1200 1200 1200 1200, clip, width=0.45\textwidth]{res/CH4/ch4_w0.png}
\includegraphics[trim=1200 1200 1200 1200, clip, width=0.45\textwidth]{res/CH4/ch4_w1.png}\\
\includegraphics[trim=1200 1200 1200 1200, clip, width=0.45\textwidth]{res/CH4/ch4_w2.png}
\includegraphics[trim=1200 1200 1200 1200, clip, width=0.45\textwidth]{res/CH4/ch4_w3.png}\\
\includegraphics[trim=1200 1200 1200 1200, clip, width=0.45\textwidth]{res/CH4/ch4_w4.png}
\caption{Die fünf besetzten Orbitale des CH$_4$\-/Moleküls,
nach aufsteigender Energie sortiert.
\textcolor{blue}{$\blacksquare$} positiv,
\textcolor{red}{$\blacksquare$} negativ.}\label{ch4_orbitals}
\end{figure}

\end{enumerate}

Diese berechneten Orbitale stimmen auch visuell mit denen aus NWChem überein
und bestätigen ebenfalls, dass die Implementierung richtig ist.

Wichtig ist noch, dass man die hier dargestellten Wellenfunktionen von deren Aufenthaltsdichten trennt.
Die Wellenfunktion eines Moleküls ist keine reale messbare Eigenschaft,
sondern ein theoretisches Konstrukt.
Die Aufenthaltsdichte hingegen kann im Experiment beobachtet werden.
Dennoch ist es üblich die Wellenfunktion-Darstellung zu verwenden.

Zum Vergleich die Aufenthaltsdichten der Orbitale des Wassermoleküls:

\begin{figure}[H]
    \centering
    \includegraphics[trim=1200 1200 1200 1200, clip, width=0.32\textwidth]{res/H2O/h2o_d0.png}
    \includegraphics[trim=1200 1200 1200 1200, clip, width=0.32\textwidth]{res/H2O/h2o_d1.png}
    \includegraphics[trim=1200 1200 1200 1200, clip, width=0.32\textwidth]{res/H2O/h2o_d2.png}\\
    \includegraphics[trim=1200 1200 1200 1200, clip, width=0.32\textwidth]{res/H2O/h2o_d3.png}
    \includegraphics[trim=1200 1200 1200 1200, clip, width=0.32\textwidth]{res/H2O/h2o_d4.png}
    \caption{Die Aufenthaltsdichten der besetzten Orbitale des H$_2$O\-/Moleküls
    in aufsteigender Energie.
    \textcolor{blue}{$\blacksquare$} positiv,
    \textcolor{red}{$\blacksquare$} negativ.}\label{h2o_densities}
\end{figure}