\usetikzlibrary{positioning}

\begin{figure}[H]
    
\tikzset{
    state/.style={
           rectangle,
           rounded corners,
           draw=black, very thick,
           minimum height=2em,
           inner sep=2pt,
           text centered,
           },
}
\begin{center}
\begin{tikzpicture}[->,>=stealth']

    \node[state] (Input)
    {
        \begin{tabular}{l}
        \textbf{Input:}\\
        \parbox{6cm}{\begin{enumerate}
            \item Anzahl der Elektronen $n$.
            \item $m$ Atomorbitale $\chi$.
            \item Molekülstruktur.
        \end{enumerate}}\\[2em]
        \end{tabular}
    };

    \node[state, below left of=Input, node distance=3.5cm] (Overlap)
    {
        \begin{tabular}{l}
        Berechne $S$\\
        mit \cref{rh_mtx_elements}.
        \end{tabular}
    };

    \node[state, below right of=Input, node distance=3.5cm] (Guess)
    {
        \begin{tabular}{l}
        Stelle eine Schätzung für\\
        die Koeffizienten in $C$ auf.
        \end{tabular}
    };

    \node[state, below of=Guess, node distance=2cm] (Fock)
    {
        \begin{tabular}{l}
        Berechne $F$\\
        mit \cref{rh_mtx_elements}.
        \end{tabular}
    };

    \node[state, below of=Overlap, node distance=3.5cm] (Roothaan)
    {
        \begin{tabular}{l}
        Löse $FC = SC\lambda$,\\
        erhalte neue $C, \lambda$.
        \end{tabular}
    };

    \node[state, below right of=Roothaan, node distance=3.5cm] (Comp)
    {
        \begin{tabular}{l}
        Vergleiche $C,\lambda$\\
        mit letzter Iteration. 
        \end{tabular}
    };

    \node[state, below right of=Fock, node distance=3.5cm] (Convergence)
    {
        Konvergiert?
    };

    \node[state, above right of=Convergence, node distance=3.5cm] (Done)
    {
        Fertig.
    };

    \path (Input) edge[] (Overlap)
                  edge[] (Guess)
        (Overlap) edge[] (Roothaan)
        (Guess)   edge[] (Fock)
        (Fock)    edge[] (Roothaan)
        (Roothaan) edge[] (Comp)
        (Comp)     edge[] (Convergence)
        (Convergence) edge[] node[right]{Ja} (Done)
                      edge[] node[left]{Nein}(Fock);
\end{tikzpicture}

\caption{Diagramm des SCF-Verfahrens.}\label{scf-chart}

\end{center}
\end{figure}