\section{Hartree-Fock}

Im Hartree-Fock-Ansatz wird aus dem Hamilton-Operator, der auf die gesamte Wellenfunktion wirkt, 
ein 1-Elektronen-Operator entwickelt, der Fock-Operator $\hat{F}$.
Da dieser Operator nur auf einzelne Elektronen wirkt, 
kann eine Pseudo-Eigenwerts-Gleichung für jedes Orbital erstellt werden.

Außerdem ist in diesem Ansatz die Elektron-Elektron-Abstoßung nur approximativ behandelt.
Dabei wird vereinfacht für die Repulsionen nur die Abstoßung die ein Elektron in einem gemittelten Potential
, das durch die anderen Elektronen entsteht, berechnet.
Für eine genauere Beschreibung müsste die Abstoßung aller Elektronen-Paare individuell berücksichtigt werden.
TODO Visualisierung.

Über die Hartree-Fock-Methode folgen diese Gleichung für jedes Spinorbital:
\begin{equation}
  \hat{F} \psi_i = \epsilon_i \psi_i \quad i = 0 \dots n
\end{equation}

\subsection{Fock Operator}


- Lösungsweg über das SCF-Verfahren (Matrix-Darstellung, ...)

- Verwendung von Basisfunktionen (Konstruktion der Wellenfunktion)

- Implementierung (größten Probleme: Integral-Evaluierung und
Matrix-Diagonalisierung)

\section{DFT}
- Herleitung (Nur die Idee/Ergebnisse, da wahrscheinlich über
meinem Niveau)

- Konkrete Umsetzung durch die Kohn-Sham-Gleichung (Terme in der
Schrödinger-Glg. + XC-Funktionale)

- Lösung durch FEM + PINVIT (+ LDA)

- Implemtierung über UG4 LUA