\section{Hartree-Fock}
\subsection{Energie}
Wir betrachten zuerst die Energie, die wir minimieren möchten.
Diese kombinieren wir dann mit unserer Darstellung 
der Gesamt\-/Wellenfunktion \cref{slater}:
\begin{flalign}
  E_\textrm{HF}[\Psi] 
    &= \langle \Psi \vert \hat{H}_{\text{el}} \vert \Psi \rangle \nonumber\\ 
    &= \langle \Psi \vert \hat{H}_{\text{core}} + \hat{V}_{ee} \vert \Psi \rangle \nonumber\\
    &= \langle \Psi \vert \hat{H}_{\text{core}} \vert \Psi \rangle 
    + \langle \Psi \vert \hat{V}_{ee} \vert \Psi \rangle &\vert \textrm{ Slater-Condon-Regel} \nonumber\\
    &= \sum_i^{2n} \langle \varphi_i \vert \hat{H}_{\text{core}} \vert \varphi_i \rangle
      + \frac{1}{2} \sum_{i, j}^{2n} \left( 
      \left[ \varphi_i \varphi_i \vert \varphi_j\varphi_j \right] 
      - \left[ \varphi_i\varphi_j \vert \varphi_j\varphi_i \right]
      \right)
\end{flalign}

\cite[S. 235, S.253]{atkins_friedman_2011}

\subsection{2\-/Elektronen\-/Integrale}
Bei den Termen 
$\left[ \varphi_i \varphi_i \vert \varphi_j\varphi_j \right]$ und
$\left[ \varphi_i \varphi_j \vert \varphi_j\varphi_i \right]$
handelt es sich um 2\-/Elektronen\-/Integrale,
welche allgemein definiert sind als:

\begin{equation}
  \left[ \varphi_i \varphi_j \vert \varphi_k \varphi_l \right] := 
  \int \varphi_i^*(1) \varphi_j(1) \frac{1}{r_{12}} \varphi_k^*(2) \varphi_l(2) \,d\tau_1 \,d\tau_2
\end{equation}

\cite[S. 19]{tc2_3}

\subsubsection*{Coulomb-Integral}
Das Coulomb\-/Integral $\left[ \varphi_i \varphi_i \vert \varphi_j\varphi_j \right]$
stellt die elektrostatische Abstoßung zwischen den Elektronen der Orbitale $\varphi_i$ und $\varphi_j$ dar:

\begin{equation}\label{coulomb}
\begin{aligned}
  \left[ \varphi_i \varphi_i \vert \varphi_j \varphi_j \right] &= 
  \int \varphi_i^*(1) \varphi_i(1) \frac{1}{r_{12}} \varphi_j^*(2) \varphi_j(2) \,d\tau_1 \,d\tau_2\\
  &= \int \varphi_i^*(1) \hat{J}_j \varphi_i(1) \,d\tau_1 \\ 
  &= \langle \varphi_i \vert \hat{J}_j \vert \varphi_i \rangle
\end{aligned}
\end{equation}

Der zugehörige Coulomb\-/Operator $\hat{J}_j$ ist definiert als:
\begin{equation}
  \hat{J}_j \varphi_i(1):= 
  \int \varphi_j^*(2) \frac{1}{r_{12}} \varphi_j(2) \varphi_i(1) \,d\tau_2 
\end{equation}

Das Coulomb\-/Integral alleine überschätzt die elektrostatische Abstoßung zweier Orbitale,
weil im Integral\-/Bild Elektronen sich beliebig nahekommen können.
Dies ignoriert jedoch das Pauli-Ausschluss-Prinzip, 
welches zwei Elektronen mit identischen quantenmechanischen Zuständen ausschließt.
Dieses verstärkte Meiden der Elektronen wird auch Pauli-Repulsion genannt.

\cite[S. 206]{lewars_2016} \cite[S. 23]{tc2_3}

\subsubsection*{Austausch-Integral}
Aus der Slater\-/Determinante, welche das Pauli-Ausschluss-Prinzip erzwingt,
geht das Austausch\-/Integral $\left[ \varphi_i \varphi_j \vert \varphi_j\varphi_i \right]$ hervor.
Dieses Integral ist ein Korrektur\-/Term für die im Coulomb\-/Integral vernachlässigte Pauli\-/Repulsion.
Es ist gegeben durch:


\begin{equation}\label{exchange}
  \begin{aligned}
  \left[ \varphi_i \varphi_j \vert \varphi_j \varphi_i \right] &= 
  \int \varphi_i^*(1) \varphi_j(1) \frac{1}{r_{12}} \varphi_j^*(2) \varphi_i(2) \,d\tau_1 \,d\tau_2\\
  &= \int \varphi_i^*(1) \hat{K}_j \varphi_i(1) \,d\tau_1 \\ 
  &= \langle \varphi_i \vert \hat{K}_j \vert \varphi_i \rangle
\end{aligned}
\end{equation}

Der zugehörige Austausch\-/Operator $\hat{K}_j$ ist definiert als:
\begin{equation}
  \hat{K}_j \varphi_i(1) :=
  \int \varphi_j^*(2) \frac{1}{r_{12}} \varphi_j(1) \varphi_i(2) \,d\tau_2
\end{equation}

\cite[S. 206]{lewars_2016} \cite[S. 23]{tc2_3}

\subsection{Herleitung der Hartree-Fock-Gleichungen}
Wir suchen nun nach einer Extremstelle für das Energie-Funktional $E_\textrm{HF}[\Psi]$ unter der Bedingung, 
dass die Spinorbitale orthonormal bleiben. 
Dafür werden im Folgenden Lagrange\-/Multiplikatoren in einem Variations\-/Verfahren verwendet.
Wir variieren beliebig im Bezug auf die Spinorbitale: 
\begin{equation}
\Psi \rightarrow \Psi + \delta \Psi
\text{, sodass alle } \varphi_i \rightarrow \varphi_i + \delta \varphi_i
\end{equation}

\subsubsection*{Energie}
Die Energie für die variierte Wellenfunktion ist dann:
\begin{equation}
\begin{aligned}
  E_\textrm{HF}[\Psi + \delta \Psi]
  &= \langle \Psi + \delta \Psi \vert \hat{H}_{\text{el}} \vert \Psi + \delta \Psi \rangle\\
  &= \langle \Psi \vert \hat{H}_{\text{el}} \vert \Psi \rangle +
  \underbrace{\langle \delta \Psi \vert \hat{H}_{\text{el}} \vert \Psi \rangle +
  \langle \Psi \vert \hat{H}_{\text{el}} \vert \delta \Psi \rangle}_\textrm{Erste Variation $\delta E_\textrm{HF}$} +
  \langle \delta \Psi \vert \hat{H}_{\text{el}} \vert \delta \Psi \rangle
\end{aligned}
\end{equation}
Mit
\begin{equation}
  \begin{split}
  \delta E_\textrm{HF} &= 
  \sum_i^{2n} \langle \delta \varphi_i \vert \hat{H}_{\text{core}} \vert \varphi_i \rangle
  + \frac{1}{2} \sum_{i, j}^{2n} \left( 
    \left[ \delta \varphi_i \varphi_i \vert \varphi_j \varphi_j \right]
  + \left[ \varphi_i \delta \varphi_i \vert \varphi_j \varphi_j \right]
  - \left[ \delta \varphi_i \varphi_j \vert \varphi_j \varphi_i \right]
  - \left[ \varphi_i \delta \varphi_j \vert \varphi_j \varphi_i \right]
  \right)\\
  &+ \sum_i^{2n} \langle \varphi_i \vert \hat{H}_{\text{core}} \vert \delta \varphi_i \rangle
  + \frac{1}{2} \sum_{i, j}^{2n} \left( 
    \left[ \varphi_i \varphi_i \vert \delta \varphi_j \varphi_j \right] 
  + \left[ \varphi_i \varphi_i \vert \varphi_j \delta \varphi_j \right]
  - \left[ \varphi_i \varphi_j \vert \delta \varphi_j \varphi_i \right]
  - \left[ \varphi_i \varphi_j \vert \varphi_j \delta \varphi_i \right]
  \right)
  \end{split}
\end{equation}
Dies vereinfacht zu
\begin{equation}\label{E_HF}
  \begin{split}
  \delta E_\textrm{HF} &=
  \sum_i^{2n} \langle \delta \varphi_i \vert \hat{H}_{\text{core}} \vert \varphi_i \rangle
  + \sum_{i, j}^{2n} \left( 
    \left[ \delta \varphi_i \varphi_i \vert \varphi_j \varphi_j \right]
  - \left[ \delta \varphi_i \varphi_j \vert \varphi_j \varphi_i \right]
  \right)\\
  &+ \underbrace{\left( \sum_i^{2n} \langle \delta \varphi_i \vert \hat{H}_{\text{core}} \vert \varphi_i \rangle
  + \sum_{i, j}^{2n} \left( 
    \left[ \delta \varphi_i \varphi_i \vert \varphi_j \varphi_j \right]
  - \left[ \delta \varphi_i \varphi_j \vert \varphi_j \varphi_i \right]
  \right)\right)^*}_\textrm{Komplexe Konjugation $\Theta_E$}\\
  &= 
  \sum_i^{2n} \langle \delta \varphi_i \vert \hat{H}_{\text{core}} \vert \varphi_i \rangle
  + \sum_{i, j}^{2n} \left( 
    \left[ \delta \varphi_i \varphi_i \vert \varphi_j \varphi_j \right]
  - \left[ \delta \varphi_i \varphi_j \vert \varphi_j \varphi_i \right]
  \right) + \Theta_E
  \end{split}
\end{equation}
Für eine Extremstelle muss diese erste Variation verschwinden, wir fordern:
\begin{equation}
  \delta E_\textrm{HF} \overset{!}{=} 0
\end{equation}

\subsubsection*{Orthonormalität}
Wir fordern ebenfalls, dass die Spinorbitale orthonormal bleiben:
\begin{equation}
  \langle \varphi_i \vert \varphi_j \rangle \overset{!}{=} \delta_{ij}\quad \forall i,j
\end{equation}

\subsubsection*{Lagrange Multiplikatoren}
Kombiniert man beide Forderungen mit Lagrange Multiplikatoren erhält man das Funktional:
\begin{equation}
  \mathcal{L}[\Phi] = E_\textrm{HF}[\Phi]
  - \sum_{i,j}^{2n} \lambda_{ij}(\langle \varphi_i \vert \varphi_j \rangle - \delta_{ij})
\end{equation}
Es gilt dabei $\lambda_{ij} = \lambda_{ji}^*$,
also die $\lambda_{ij}$ bilden eine hermitische Matrix. \cite[3.40]{szabo_ostlund_1996}

Eine Extremstelle in $\mathcal{L}$ bedeutet auch eine Extremstelle in $E_\textrm{HF}$
unter der Orthonormalitäts\-/Bedingung. Also setzten wir die Erste Variation in $\mathcal{L}$ null:
\begin{equation}
  \delta \mathcal{L} = \delta E_\textrm{HF}
  - \sum^{2n}_{i,j} \lambda_{ij} \delta \langle \varphi_i \vert \varphi_j \rangle = 0
\end{equation}

Wir packen im Folgenden alle komplex konjugierten Terme in $\Theta$,
dabei gleicht das $\Theta^*$ stets dem restlichen Term.
\begin{align*}
  0 &= \delta E_\textrm{HF} 
  - \sum^{2n}_{i,j} \lambda_{ij} \delta \langle \varphi_i \vert \varphi_j \rangle\\
  &= \delta E_\textrm{HF}
  - \sum^{2n}_{i,j} \lambda_{ij} \langle \delta \varphi_i \vert \varphi_j \rangle
  - \sum^{2n}_{i,j} \lambda_{ij} \langle \varphi_i \vert \delta \varphi_j \rangle\\
  &= \delta E_\textrm{HF}
  - \sum^{2n}_{i,j} \lambda_{ij} \langle \delta \varphi_i \vert \varphi_j \rangle
  - \left(\sum^{2n}_{i,j} \lambda_{ji} \langle \delta \varphi_i \vert \varphi_j \rangle \right)^*
  &| &\textrm{ \cref{E_HF}}\\
  &=  \sum_i^{2n} \langle \delta \varphi_i \vert \hat{H}_{\text{core}} \vert \varphi_i \rangle
  + \sum_{i, j}^{2n} \left( 
    \left[ \delta \varphi_i \varphi_i \vert \varphi_j\varphi_j \right] 
    - \left[ \delta \varphi_i\varphi_j \vert \varphi_j\varphi_i \right]
  \right) 
  - \sum_{i,j}^{2n} \lambda_{ij} \langle \delta \varphi_i \vert \varphi_j \rangle + \Theta
  &| &\textrm{ \cref{coulomb}, \cref{exchange}}\\
  &=  \sum_i^{2n} \langle \delta \varphi_i \vert \hat{H}_{\text{core}} \vert \varphi_i \rangle
  + \sum_{i, j}^{2n} \left( 
    \langle \delta \varphi_i \vert \hat{J}_j \vert \varphi_i \rangle
    - \langle \delta \varphi_i \vert \hat{K}_j \vert \varphi_i \rangle
  \right) 
  - \sum_{i,j}^{2n} \lambda_{ij} \langle \delta \varphi_i \vert \varphi_j \rangle + \Theta
\end{align*}

Wir faktorisiern nun die Summe $\sum_i^{2n}$ und das $\langle \delta \varphi_i \vert$ aus:
\begin{equation*}
  \sum_i^{2n} \langle \delta \varphi_i \vert \left(
    \hat{H}_{\text{core}} \vert \varphi_i \rangle
  + \sum_j^{2n} \left(
    \hat{J}_j \vert \varphi_i \rangle
    - \hat{K}_j \vert \varphi_i \rangle
  \right) 
  - \sum_j^{2n}\lambda_{ij} \vert \varphi_j \rangle \right) + \Theta
  = 0
\end{equation*}

Da jedes $\delta \varphi_i^*$ beliebig variiert werden kann,
muss der Term in der Klammer für jedes $i$ jeweils null sein für eine Nullstelle.
Wir erhalten die Hartree\-/Fock\-/Gleichungen:


\begin{align}
  \left(\hat{H}_{\text{core}} + \sum_j^{2n} 
  \left( \hat{J}_j - \hat{K}_j \right)\right)\vert\varphi_i\rangle &= 
  \sum_j^{2n}\lambda_{ij} \vert\varphi_j\rangle \nonumber\\
  \hat{F}\varphi_i &= \sum_j^{2n}\lambda_{ij} \varphi_j, \quad \forall i = 0 \dots 2n
\end{align}

Durch Matrix-Diagonalisierung erhält man die kanonischen Hartree\-/Fock\-/Gleichungen:
\begin{equation} \label{UHF}
  \hat{F} \varphi_i' = \lambda_i \varphi_i', \quad i = 0 \dots 2n
\end{equation}
Da die $\lambda_{ij}$ hermitisch sind, ist immer eine Matrix\-/Diagonalisierung möglich.
Dabei ist der Fock\-/Operator $\hat{F}$ invariant bezüglich dieser Diagonalisierung
\cite[3.64]{szabo_ostlund_1996}.

\cite[S. 253]{atkins_friedman_2011}
\cite[S. 115-119]{szabo_ostlund_1996}

\subsection{Entfernen des Spins}
Es ist möglich den Spin aus unserer Hatree\-/Fock\-/Gleichung zu entfernen,
da wir lediglich Möleküle betrachten,
dessen Orbitale stets zweifach besetzt sind mit je einem $\alpha$ und $\beta$ Elektron.
Wir müssen nur für ein beliebiges $\varphi_i$ aus \cref{UHF} umformen:
\begin{align*}
  \hat{F} \varphi_i &= \lambda_i \varphi_i \\
  \hat{F} \psi_i \alpha &= \lambda_i \psi_i \alpha
\end{align*}

Wir integrieren nun beide Seiten der Gleichung,
multipliziert mit $\alpha^*$, über die Spinkoordinate $\omega$.
Wir multiplizieren mit $\langle \alpha \vert$:
\begin{align*}
  \langle \alpha \vert \hat{F} \psi_i \vert \alpha \rangle &= \lambda_i \psi_i \vert \alpha \rangle\\
  \langle \alpha \vert \left(\hat{H}_{\text{core}} + \sum_j^{2n} 
  \left( \hat{J}_j - \hat{K}_j \right)\right) \psi_i \vert \alpha \rangle &= 
  \langle \alpha \vert \lambda_i \psi_i \vert \alpha \rangle
\end{align*}

$\hat{H}_{\text{core}}, \psi_i$ und $\lambda_i$ sind von den Spinkoordinaten unabhängig und
können außerhalb der der Spin-Integrale stehen:
\begin{align}
  \hat{H}_{\text{core}} \psi_i \underbrace{\langle \alpha \vert \alpha \rangle}_{=1} +
  \langle \alpha \vert \sum_j^{2n} 
  \left( \hat{J}_j - \hat{K}_j \right) \vert \alpha \rangle \psi_i 
  &= \lambda_i \psi_i \underbrace{\langle \alpha \vert \alpha \rangle}_{=1} \nonumber\\
  \hat{H}_{\text{core}} \psi_i + \langle \alpha \vert \sum_j^{2n}
  \left( \hat{J}_j - \hat{K}_j \right) \vert \alpha \rangle \psi_i 
  &= \lambda_i \psi_i
\end{align}

Betrachten wir nun den Term für die Elektron\-/Elektron\-/Interaktionen.
Da im Coulomb- und Austausch\-/Operator Spin\-/Orbital\-/Funktion vorkommen und
jedes Orbital zwei Elektronen mit unterschiedlichem Spin enthält, lässt sich die Summe umschreiben zu:
\begin{align*}
  \langle \alpha \vert \sum_j^{2n} \left( \hat{J}_j - \hat{K}_j \right) \vert \alpha \rangle
  &= \sum_j^{2n} \langle \alpha \vert \hat{J}_j \vert \alpha \rangle
  - \langle \alpha \vert \hat{K}_j \vert \alpha \rangle \\
  &= \sum_j^{n} 
  \langle \alpha(1) \vert \langle \alpha(2) \vert \hat{J'}_j \vert \alpha(2) \rangle \vert \alpha(1) \rangle
  + \langle \alpha(1) \vert \langle \beta(2) \vert \hat{J'}_j \vert \beta(2) \rangle \vert \alpha(1) \rangle\\
  &- \langle \alpha(1) \vert \langle \alpha(2) \vert \hat{K'}_j \vert \alpha(1) \rangle \vert \alpha(2) \rangle
  - \langle \alpha(1) \vert \langle \beta(2) \vert \hat{K'}_j \vert \beta(1) \rangle \vert \alpha(2) \rangle
\end{align*}
Wir haben die Terme der Summe mit gleichem Raumorbital kombiniert und
das Spin\-/Integral der Operatoren aus diesen herausgezogen.
Also hängen die modifizierten Operatorn $\hat{J'}_j$ und $\hat{K'}_j$ nicht von $\omega_1$ oder $\omega_2$ ab.
Wir können die Operatoren aus den Spin\-/Integralen entfernen:
\begin{align*}
  &=\sum_j^{n} 
  \hat{J'}_j \langle \alpha(1) \vert \langle \alpha(2) \vert \alpha(2) \rangle \vert \alpha(1) \rangle
  + \hat{J'}_j \langle \alpha(1) \vert \langle \beta(2) \vert \beta(2) \rangle \vert \alpha(1) \rangle\\
  &- \hat{K'}_j \langle \alpha(1) \vert \langle \alpha(2) \vert \alpha(1) \rangle \vert \alpha(2) \rangle
  - \hat{K'}_j \langle \alpha(1) \vert \langle \beta(2) \vert \beta(1) \rangle \vert \alpha(2) \rangle\\
  &=\sum_j^{n} 
  \hat{J'}_j \underbrace{\langle \alpha(1) \vert \alpha(1) \rangle}_{=1}
  \underbrace{\langle \alpha(2) \vert \alpha(2) \rangle}_{=1}
  + \hat{J'}_j \underbrace{\langle \alpha(1) \vert \alpha(1) \rangle}_{=1}
  \underbrace{\langle \beta(2) \vert \beta(2) \rangle}_{=1}\\
  &- \hat{K'}_j \underbrace{\langle \alpha(1) \vert \alpha(1) \rangle}_{=1}
  \underbrace{\langle \alpha(2) \vert \alpha(2) \rangle}_{=1}
  - \hat{K'}_j \underbrace{\langle \alpha(1) \vert \beta(1) \rangle}_{=0}
  \underbrace{\langle \beta(2) \vert \alpha(2) \rangle}_{=0}\\
  &=\sum_j^{n} 2\hat{J'}_j - \hat{K'}_j
\end{align*}

Setzen wir den letzten Term wieder in die urspüngliche Gleichung ein, erhalten wir:
\begin{equation}\label{RHF}
  \left( \hat{H}_{\text{core}} + \sum_j^{n}
  \left( 2\hat{J'}_j - \hat{K'}_j \right) \right) \psi_i 
  = \lambda_i \psi_i
\end{equation}
Das ist die Hartree\-/Fock\-/Gleichung unter der Bedingung, dass jedes Orbital doppelt besetzt ist.
Diesen Fall \cref{RHF} nennt man Restricted Hartree\-/Fock (RHF) und
den Fall mit Spin\-/Orbitalen aus \cref{UHF} Unrestricted Hartree\-/Fock (UHF).

Wir beschränken uns nur auf Restricted Hartree\-/Fock, wie bereits festgelegt.

\section{Roothaan-Hall}
Die Matrix\-/Form der Roothaan\-/Hall\-/Gleichungen:
\begin{equation} \label{roothaan}
  FC = SC\epsilon
\end{equation}

\begin{comment}
Im Hartree-Fock-Ansatz wird aus dem Hamilton-Operator, der auf die gesamte Wellenfunktion wirkt, 
ein 1-Elektronen-Operator entwickelt, der Fock-Operator $\hat{F}$.
Da dieser Operator nur auf einzelne Elektronen wirkt, 
kann eine Pseudo-Eigenwerts-Gleichung für jedes Orbital erstellt werden.

Außerdem ist in diesem Ansatz die Elektron-Elektron-Abstoßung nur approximativ behandelt.
Dabei wird vereinfacht für die Repulsionen nur die Abstoßung die ein Elektron in einem gemittelten Potential
, das durch die anderen Elektronen entsteht, berechnet.
Für eine genauere Beschreibung müsste die Abstoßung aller Elektronen-Paare individuell berücksichtigt werden.
TODO Visualisierung.

\subsection{Fock Operator}
- Lösungsweg über das SCF-Verfahren (Matrix-Darstellung, ...)

- Verwendung von Basisfunktionen (Konstruktion der Wellenfunktion)

- Implementierung (größten Probleme: Integral-Evaluierung und
Matrix-Diagonalisierung)

\section{DFT}
- Herleitung (Nur die Idee/Ergebnisse, da wahrscheinlich über
meinem Niveau)

- Konkrete Umsetzung durch die Kohn-Sham-Gleichung (Terme in der
Schrödinger-Glg. + XC-Funktionale)

- Lösung durch FEM + PINVIT (+ LDA)

- Implemtierung über UG4 LUA

Über die Hartree\-/Fock\-/Methode folgen diese Gleichung für jedes Spinorbital:

\end{comment}