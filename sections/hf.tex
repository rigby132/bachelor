\section{Hartree-Fock}
\subsection{Energie}
Wir betrachten zuerst die Energie, die wir minimieren möchten.
Diese kombinieren wir dann mit unserer Darstellung 
der Gesamt\-/Wellenfunktion \cref{slater}:
\begin{flalign}
  E_\text{HF} &= \langle \Psi \vert \hat{H}_{\text{el}} \vert \Psi \rangle \nonumber\\ 
                &= \langle \Psi \vert \hat{H}_{\text{core}} + \hat{V}_{ee} \vert \Psi \rangle \nonumber\\
                &= \langle \Psi \vert \hat{H}_{\text{core}} \vert \Psi \rangle 
                + \langle \Psi \vert \hat{V}_{ee} \vert \Psi \rangle &\vert \text{ Slater-Condon-Regel} \nonumber\\
                %&= \sum_i^{2n} \langle \varphi_i \vert \hat{H}_{\text{core}} \vert \varphi_i \rangle
                %+ \frac{1}{2} \sum_{i\neq j}^{2n} \left( 
                %  \langle \varphi_i(1) \varphi_j(2) \vert \tfrac{1}{r_{ij}} \vert \varphi_i(1) \varphi_j(2) \rangle 
                %- \langle \varphi_i(1) \varphi_j(2) \vert \tfrac{1}{r_{ij}} \vert \varphi_j(1) \varphi_i(2) \rangle 
                %\right)\\
                &= \sum_i^{2n} \langle \varphi_i \vert \hat{H}_{\text{core}} \vert \varphi_i \rangle
                + \frac{1}{2} \sum_{i, j}^{2n} \left( 
                  \left[ \varphi_i \varphi_i \vert \varphi_j\varphi_j \right] 
                  - \left[ \varphi_i\varphi_j \vert \varphi_j\varphi_i \right]
                \right)
\end{flalign}

\cite[S. 235, S.253]{atkins_friedman_2011}

\subsection{2\-/Elektronen\-/Integrale}
Bei den Termen 
$\left[ \varphi_i \varphi_i \vert \varphi_j\varphi_j \right]$ und
$\left[ \varphi_i \varphi_j \vert \varphi_j\varphi_i \right]$
handelt es sich um 2\-/Elektronen\-/Integrale,
welche allgemein definiert sind als:

\begin{equation}
  \left[ \varphi_i \varphi_j \vert \varphi_k \varphi_l \right] := 
  \int \varphi_i^*(1) \varphi_j(1) \frac{1}{r_{12}} \varphi_k^*(2) \varphi_l(2) \,d\tau_1 \,d\tau_2
\end{equation}

\cite[S. 19]{tc2_3}

\subsubsection*{Coulomb-Integral}
Das Coulomb\-/Integral $\left[ \varphi_i \varphi_i \vert \varphi_j\varphi_j \right]$
stellt die elektrostatische Abstoßung zwischen den Elektronen der Orbitale $\varphi_i$ und $\varphi_j$ dar:

\begin{align*}
  \left[ \varphi_i \varphi_i \vert \varphi_j \varphi_j \right] &= 
  \int \varphi_i^*(1) \varphi_i(1) \frac{1}{r_{12}} \varphi_j^*(2) \varphi_j(2) \,d\tau_1 \,d\tau_2\\
  &= \int \varphi_i^*(1) \hat{J}_j \varphi_i(1) \,d\tau_1 \\ 
  &= \langle \varphi_i \vert \hat{J}_j \vert \varphi_i \rangle
\end{align*}

Der zugehörige Coulomb\-/Operator $\hat{J}_j$ ist definiert als:
\begin{equation}
  \hat{J}_j \varphi_i(1):= 
  \int \varphi_j^*(2) \frac{1}{r_{12}} \varphi_j(2) \varphi_i(1) \,d\tau_2 
\end{equation}

Das Coulomb\-/Integral alleine überschätzt die elektrostatische Abstoßung zweier Orbitale,
weil im Integral\-/Bild Elektronen sich beliebig nahekommen können.
Dies ignoriert jedoch das Pauli-Ausschluss-Prinzip, 
welches zwei Elektronen mit identischem quantenmechanischem Zustand ausschließt.
Dieses verstärkte Meiden der Elektronen wird auch Pauli-Repulsion genannt.

\cite[S. 206]{lewars_2016} \cite[S. 23]{tc2_3}

\subsubsection*{Austausch-Integral}
Aus der Slater\-/Determinante, welche das Pauli-Ausschluss-Prinzip erzwingt,
geht das Austausch\-/Integral $\left[ \varphi_i \varphi_j \vert \varphi_j\varphi_i \right]$ hervor.
Dieses Integral ist ein Korrektur\-/Term für die im Coulomb\-/Integral vernachlässigte Pauli\-/Repulsion.
Es ist gegeben durch:

\begin{align*}
  \left[ \varphi_i \varphi_j \vert \varphi_j \varphi_i \right] &= 
  \int \varphi_i^*(1) \varphi_j(1) \frac{1}{r_{12}} \varphi_j^*(2) \varphi_i(2) \,d\tau_1 \,d\tau_2\\
  &= \int \varphi_i^*(1) \hat{K}_j \varphi_i(1) \,d\tau_1 \\ 
  &= \langle \varphi_i \vert \hat{K}_j \vert \varphi_i \rangle
\end{align*}

Der zugehörige Austausch\-/Operator $\hat{K}_j$ ist definiert als:
\begin{align}
  \hat{K}_j \varphi_i(1)&:= 
  \int \varphi_j^*(2) \frac{1}{r_{12}} \hat{P}_{12} \varphi_j(2) \varphi_i(1) \,d\tau_2 \\
  &= \int \varphi_j^*(2) \frac{1}{r_{12}} \varphi_j(1) \varphi_i(2) \,d\tau_2 \nonumber
\end{align}

Der Permutations-Operator $\hat{P}_{12}$ vertauscht die Elektronen der betroffenen Orbitale.

\cite[S. 206]{lewars_2016} \cite[S. 23]{tc2_3}

\subsection{Herleitung der Hartree-Fock-Gleichungen}
Wir suchen nun nach einer Extremstelle für den Energie-Term $E_\text{HF}$ unter der Bedingung, 
dass die Spinorbitale orthonormal bleiben. 
Dafür werden im Folgenden Lagrange\-/Multiplikatoren in einem Variations\-/Verfahren verwendet.
Wir variieren bezüglich der $\varphi^*$ und erhalten die Gleichung:

\begin{equation*}
  \sum_i^{2n} \langle \delta \varphi_i \vert \hat{H}_{\text{core}} \vert \varphi_i \rangle
  + \sum_{i, j}^{2n} \left( 
    \left[ \delta \varphi_i \varphi_i \vert \varphi_j\varphi_j \right] 
    - \left[ \delta \varphi_i\varphi_j \vert \varphi_j\varphi_i \right]
  \right) 
  - \sum_{i,j}^{2n} \lambda_{ij} \langle \delta \varphi_i \vert \varphi_j \rangle
  = 0
\end{equation*}

Wir faktorisiern nun die Summe $\sum_i^{2n}$ und das $\langle \delta \varphi_i \vert$ aus:

\begin{equation*}
  \sum_i^{2n} \langle \delta \varphi_i \vert \left(
    \hat{H}_{\text{core}} \vert \varphi_i \rangle
  + \sum_j^{2n} \left(
    \hat{J}_j \vert \varphi_i \rangle
    - \hat{K}_j \vert \varphi_i \rangle
  \right) 
  - \sum_j^{2n}\lambda_{ij} \vert \varphi_j \rangle \right)
  = 0
\end{equation*}

Da jedes $\delta \varphi_i^*$ beliebig variiert werden kann, muss der Term in der Klammer für jedes $i$ jeweils null sein.
Wir erhalten die Hartree\-/Fock\-/Gleichungen:


\begin{align} \label{uhf}
  \left(\hat{H}_{\text{core}} + \sum_j^{2n} \left( \hat{J}_j - \hat{K}_j \right)\right)\varphi_i &= 
  \sum_j^{2n}\lambda_{ij} \varphi_j \nonumber\\
  \hat{F}\varphi_i &= \sum_j^{2n}\lambda_{ij} \varphi_j, \quad \forall i = 0 \dots 2n
\end{align}

Durch Matrix-Diagonalisierung erhält man die kanonischen Hartree\-/Fock\-/Gleichungen:

\begin{equation} \label{uhf_can}
  \hat{F} \varphi_i' = \varepsilon_i \varphi_i', \quad i = 0 \dots 2n
\end{equation}

\cite[S. 253]{atkins_friedman_2011}

TODO Spin ausintegrieren.

\section{Roothaan-Hall}
Die Matrix\-/Form der Roothaan\-/Hall\-/Gleichungen:
\begin{equation} \label{roothaan}
  FC = SC\epsilon
\end{equation}

\begin{comment}
Im Hartree-Fock-Ansatz wird aus dem Hamilton-Operator, der auf die gesamte Wellenfunktion wirkt, 
ein 1-Elektronen-Operator entwickelt, der Fock-Operator $\hat{F}$.
Da dieser Operator nur auf einzelne Elektronen wirkt, 
kann eine Pseudo-Eigenwerts-Gleichung für jedes Orbital erstellt werden.

Außerdem ist in diesem Ansatz die Elektron-Elektron-Abstoßung nur approximativ behandelt.
Dabei wird vereinfacht für die Repulsionen nur die Abstoßung die ein Elektron in einem gemittelten Potential
, das durch die anderen Elektronen entsteht, berechnet.
Für eine genauere Beschreibung müsste die Abstoßung aller Elektronen-Paare individuell berücksichtigt werden.
TODO Visualisierung.

\subsection{Fock Operator}
- Lösungsweg über das SCF-Verfahren (Matrix-Darstellung, ...)

- Verwendung von Basisfunktionen (Konstruktion der Wellenfunktion)

- Implementierung (größten Probleme: Integral-Evaluierung und
Matrix-Diagonalisierung)

\section{DFT}
- Herleitung (Nur die Idee/Ergebnisse, da wahrscheinlich über
meinem Niveau)

- Konkrete Umsetzung durch die Kohn-Sham-Gleichung (Terme in der
Schrödinger-Glg. + XC-Funktionale)

- Lösung durch FEM + PINVIT (+ LDA)

- Implemtierung über UG4 LUA

Über die Hartree\-/Fock\-/Methode folgen diese Gleichung für jedes Spinorbital:

\end{comment}