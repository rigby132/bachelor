\section{Hartree-Fock}
Wir betrachten zuerst die Energie, die wir minimieren möchten.
Diese kombinieren wir dann mit unserer Darstellung 
der Gesamt\-/Wellenfunktion \cref{slater}:
\begin{flalign*}
  E_{\text{HF}} &= \langle \Psi \vert \hat{H}_{\text{el}} \vert \Psi \rangle \\ 
                &= \langle \Psi \vert \hat{H}_{\text{core}} + \hat{V}_{ee} \vert \Psi \rangle \\
                &= \langle \Psi \vert \hat{H}_{\text{core}} \vert \Psi \rangle 
                + \langle \Psi \vert \hat{V}_{ee} \vert \Psi \rangle &\vert \text{ Slater-Condon-Regel} \\
                %&= \sum_i^{2n} \langle \varphi_i \vert \hat{H}_{\text{core}} \vert \varphi_i \rangle
                %+ \frac{1}{2} \sum_{i\neq j}^{2n} \left( 
                %  \langle \varphi_i(1) \varphi_j(2) \vert \tfrac{1}{r_{ij}} \vert \varphi_i(1) \varphi_j(2) \rangle 
                %- \langle \varphi_i(1) \varphi_j(2) \vert \tfrac{1}{r_{ij}} \vert \varphi_j(1) \varphi_i(2) \rangle 
                %\right)\\
                &= \sum_i^{2n} \langle \varphi_i \vert \hat{H}_{\text{core}} \vert \varphi_i \rangle
                + \frac{1}{2} \sum_{i\neq j}^{2n} \left( 
                  \left[ \varphi_i \varphi_i \vert \varphi_j\varphi_j \right] 
                  - \left[ \varphi_i\varphi_j \vert \varphi_j\varphi_i \right]
                \right)\\
\end{flalign*}
TODO coulomb und austausch integrale ausschreiben + cite + Terme erklären

Nach Anwendung von Lagrange\-/Multiplikatoren 
(mit Bedingung $\langle \varphi_i \vert \varphi_j \rangle = \delta_{ij})$ erhalten wir
die Hartree\-/Fock\-/Gleichungen:

\begin{equation} \label{uhf}
  \hat{F}(1) \varphi_i(1) = \sum_j^{2n}\lambda_{ij} \varphi_j(1), \quad i = 0 \dots 2n
\end{equation}

Durch Matrix-Diagonalisierung erhält man die kanonischen Hartree\-/Fock\-/Gleichungen:

\begin{equation} \label{uhf_can}
  \hat{F}(1) \varphi_i'(1) = \epsilon_i \varphi_i'(1), \quad i = 0 \dots 2n
\end{equation}

Wobei der Fock\-/Operator $\hat{F}$ gegeben ist durch:

\begin{equation} \label{fockoperator}
  \hat{F} = \hat{H}_{\text{core}} + \sum_j^{2n} \left( \hat{J}_j - \hat{K}_j \right)
\end{equation}
TODO zeigen was J und K sind + Spin ausintegrieren. bei Operatoren (1) weglassen?

\section{Roothaan-Hall}
Die Matrix\-/Form der Roothaan\-/Hall\-/Gleichungen:
\begin{equation} \label{roothaan}
  FC = SC\epsilon
\end{equation}

\begin{comment}
Im Hartree-Fock-Ansatz wird aus dem Hamilton-Operator, der auf die gesamte Wellenfunktion wirkt, 
ein 1-Elektronen-Operator entwickelt, der Fock-Operator $\hat{F}$.
Da dieser Operator nur auf einzelne Elektronen wirkt, 
kann eine Pseudo-Eigenwerts-Gleichung für jedes Orbital erstellt werden.

Außerdem ist in diesem Ansatz die Elektron-Elektron-Abstoßung nur approximativ behandelt.
Dabei wird vereinfacht für die Repulsionen nur die Abstoßung die ein Elektron in einem gemittelten Potential
, das durch die anderen Elektronen entsteht, berechnet.
Für eine genauere Beschreibung müsste die Abstoßung aller Elektronen-Paare individuell berücksichtigt werden.
TODO Visualisierung.

\subsection{Fock Operator}
- Lösungsweg über das SCF-Verfahren (Matrix-Darstellung, ...)

- Verwendung von Basisfunktionen (Konstruktion der Wellenfunktion)

- Implementierung (größten Probleme: Integral-Evaluierung und
Matrix-Diagonalisierung)

\section{DFT}
- Herleitung (Nur die Idee/Ergebnisse, da wahrscheinlich über
meinem Niveau)

- Konkrete Umsetzung durch die Kohn-Sham-Gleichung (Terme in der
Schrödinger-Glg. + XC-Funktionale)

- Lösung durch FEM + PINVIT (+ LDA)

- Implemtierung über UG4 LUA

Über die Hartree\-/Fock\-/Methode folgen diese Gleichung für jedes Spinorbital:

\end{comment}