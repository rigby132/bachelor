\chapter{Ausblick}

In dieser Arbeit wurde mithilfe der dargelegten HF-Theorie
erfolgreich ein Programm zur qualitativen Berechnung von Molekül\-/Energien
implementiert. Es bieten sich nun einige Optionen zur Weiterführung dieses Projektes an.

\section{Erweiterung der Funktionalitäten}
Momentan lassen sich nur Hartree\-/Fock\-/Energien berechnen und Orbitale exportieren.
Jedoch gibt es eine Vielzahl an weiteren Funktionalitäten oder Ergebnissen, die nützlich sein können.
Zu diesen gehören unter anderem:
\begin{enumerate}
    \item Berechnungen mit der unrestricted Hartree-Fock-Methode (UHF):\\
            Wie bei \cref{RHF} bereits erwähnt, können mit diesem Ansatz 
            nur Moleküle mit vollständig besetzten Orbitalen behandelt werden (restricted HF).
            Mit der Erweiterung zu UHF wären Berechnungen möglich
            mit nur halb besetzten Orbitalen, welche ebenfalls stabile
            Lösungen zulassen. 
            Dazu müsste man jedoch den Spin beibehalten
            und nicht wie in \cref*{remove-spin-section} entfernen.
    \item Berechnung weiterer Energien:\\
            Es gibt neben der Hartree-Fock-Energie noch
            viele weitere Energien die berechnet werden können.
            Einer dieser ist z.B. die Grundzustandsenergie, welche die Energie
            eines Moleküls bei 0 Kelvin angibt. \cite[S. 40]{lewars_2016}
    \item Geometrie-Optimierungen:\\
            In der derzeitigen Implementierung wird als Eingabe
            eine Molekülgeometrie (Positionen aller Atomkerne) erwartet.
            Jedoch gibt es ausgehend von dieser Eingabe keine automatisierte Möglichkeit
            eine Geometrie zu erlangen, welche die HF-Energie des Moleküls minimiert.
    \item Schwingungs-Frequenzen:\\
            Das sind Frequenzen periodischer Bewegungen innerhalb eines Moleküls
            und finden verschiedene Anwendungen. Diese Frequenzen werden benötigt,
            um die erwähnte Grundzustandsenergie zu berechnen und können auch bei einer
            Geometrie-Optimierung helfen. Außerdem lassen sich dadurch auch
            Infrarot-Spektren generieren.\cite[2.5, 5.5.3]{lewars_2016}
\end{enumerate}

\section{Verbesserung der Hartree-Fock Implementierung}
Wie bereits erwähnt, kann die entwickelte Software qualitativ
Ergebnisse produzieren. Jedoch gibt es eine Vielzahl an
möglichen Implementierungen der Theorie.
Die Optimierung über das SCF-Verfahren kann auch durch
andere herkömmliche Methoden umgesetzt werden. Außerdem gibt
es auch mehrere Wege die 2\-/Elektronen\-/Integrale zu berechnen,
hier kann ebenfalls ein alternatives Schema gewählt werden.
Man könnte auch andere Basisfunktionen verwenden statt der Atomorbitale.
Dies sind nur einige Beispiele für alternative Implementierungen
innerhalb der HF\-/Theorie. TODO cite

\section{Post-Hartree-Fock}
Die Energien in der HF-Theorie überschätzen die Abstoßung der Elektronen untereinander,
weshalb Weiterentwicklungen der HF-Theorie erstellt wurden,
welche die Elektronen-Interaktionen genauer behandeln.
Zu diesen gehören Methoden wie das Møller\-/Plesset- (MP)
und Configuration\-/Interaction\-/Verfahren (CI).
Diese erweitern die HF-Theorie um weitere Terme,
welche Elektronen aus besetzten Orbitalen in virtuelle Orbitale setzen.
Durch das Einbinden dieser zusätzlichen Zustände lassen
sich die Elektronen\-/Repulsionen exakter beschreiben.

\cite[S. 3]{tc2_phf}
\cite[5.4.2, 5.4.3]{lewars_2016}

\section{Optimierung}
Wie in \cref{implementation-section} zu sehen ist,
wurden wichtige und naheliegende Optimierungen
sowohl auf theoretischer als auch auf programmier-technischer Seite bereits durchgeführt.
Diese sind besonders auf die Berechnung der 2\-/Elektronen\-/Integrale fokusiert,
da dieser Teil bei Weitem die meiste Rechenzeit erfordert.

In der zu dieser Arbeit begleitenden Software handelt es sich
um eine direkte Implementierung der besprochenen Theorie.
Um die Einfachheit dieser Implementierung zu wahren, wurde auf den Einsatz von
fortgeschrittenen Optimierungsstrategien verzichtet.
Diese könnten jedoch in weiterführenden Implementierungen eingeführt werden,
wodurch auch die Berechnung großer molekularer System möglich wäre.

Im theoretischen Bereich bietet sich dazu die Umsetzung der Fast-Multipole-Method an,
in welcher Interaktionen zwischen relativ weit entfernten Atomen
je nach benötigter Präzision systematisch approximiert werden.
Außerdem sind noch Laufzeit-Verbesserungen durch das Ausnutzen von Molekül-Symmetrien möglich,
diese erlauben z.B. das Wiederverwenden von einzelnen berechneten Werten.

Neben der auf Theorie basierenden Optimierungen können ebenfalls Verbesserungen
durch effektivere Verwendung der Computerhardware erzielt werden.
Die Einführung cache-freundlicher Speicherstrukturen sowie das Meiden redundanter
Berechnungen ist deshalb besonders naheliegend.
Zudem können zusätzliche Verfahren für das parallele Berechnen der HF-Methode eingebaut werden,
zu diesen gehören Optionen wie das Rechnen auf GPUs mit OpenCL\cite{opencl}
oder sogar Cluster-Systemen über z.B. MPI\cite{mpi}.

Erst durch die Verwendung dieser fortgeschrittenen Optimierungen sollte es möglich sein
vergleichbare Laufzeiten zu denen aus professionellen Programmen zu erzielen.
