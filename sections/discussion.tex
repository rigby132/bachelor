\chapter{Diskussion/Ausblick}

\section{Einordnung der Hartee\-/Fock\-/Theorie in der Chemie}
Die verwendete HF\-/Theorie gehört zur Klasse der Ab\-/Initio\-/Methoden,
welche versuchen nur basierend auf theoretischen Modellen und
universellen Konstanten Berechnung durchzuführen.
Diese Verfahren stehen im Gegensatz zu den Semiempirischen\-/Methoden,
welche zusätzlich experimentell bestimmte Werte in die Berechnungen einbinden.

Die HF-Theorie ist der einfachste brauchbare Vertreter
der Klasse der Ab\-/Initio\-/Methoden
und wird deshalb bei entweder sonst zu großen System oder
als Start-Approximation für sehr akkurate Berechnungen verwendet.
\cite[S. 433]{structure_2013}

Basierend auf der HF-Theorie existieren die Post\-/Hartree\-/Fock\-/Methoden,
diese verbesseren die Genauigkeit der Ergebnisse hauptsächlich
durch eine genauere Behandlung der Elektron-Elektron-Interaktionen.

Neben den Post\-/HF\-/Methoden existieren zudem Verfahren
mit unterschiedlichen Ansätzen. Eine der relevantesten ist
die Dichte\-/Funktional\-/Theorie (DFT), welche nicht auf
Wellenfunktionen aufbaut sondern direkt mit den
Dichtefunktionen der Elektronen arbeitet.

\section{Ausblick}
In dieser Arbeit wurde mithilfe der dargelegten HF-Theorie
erfolgreich ein Programm zur qualitativen Berechnung von Molekül\-/Grundzustands\-/Energien
implementiert. Es bieten sich nun drei Optionen zur Weiterführung diese Projektes an:
\begin{enumerate}
    \item Mehr Ergebnisse durch Einführung zusätzlicher Werkzeuge.\\
    Momentan lassen sich nur die Grundzustandsenergien berechnen und Orbitale exportieren.
    Jedoch gibt es eine Vielzahl an weiteren Werkzeugen, die nützlich sein können.
    Zu diesen gehört zum Beispiel die Geometrie-Optimierung durch das Programm. 
    \item Genauere Ergebnisse durch Erweiterung der entwickelten HF-Theorie.\\
    Wie bereits erwähnt, kann die entwickelte Software 
    lediglich qualitativ belastbare Ergebnisse produzieren.
    Durch Verwendung zusätzlicher Methoden wären dann
    quantitativ brauchbare Berechnungen möglich.
    \item Schnellere Ergebnisse durch Optimierungen.\\
    Beim Berechnen der Resultate ist deutlich geworden,
    dass diese Implementierung noch vergleichsweise sehr inneffizient ist.
    Vor allem beim Evaluieren der 2\-/Elektronen\-/Integrale besteht hoher Optimierungsbedarf.
\end{enumerate}