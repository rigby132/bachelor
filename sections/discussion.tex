\chapter{Ausblick}

In dieser Arbeit wurde mithilfe der dargelegten HF-Theorie
erfolgreich ein Programm zur qualitativen Berechnung von Molekül\-/Energien
implementiert. Es bieten sich nun einige Optionen zur Weiterführung diese Projektes an.

\section{Erweiterung der Funktionalitäten}
Momentan lassen sich nur die Hartree\-/Fock\-/Energie berechnen und Orbitale exportieren.
Jedoch gibt es eine Vielzahl an weiteren Funktionalitäten, die nützlich sein können.
Zu diesen gehört zum Beispiel die Geometrie-Optimierung durch das Programm. 

\section{Quantitative Berechnungen}
Wie bereits erwähnt, kann die entwickelte Software 
lediglich qualitativ belastbare Ergebnisse produzieren.
Durch Verwendung zusätzlicher Methoden wären
quantitativ brauchbare Berechnungen möglich.

\section{Post-Hartree-Fock}
Wie in \cref{posthf} erwähnt, gibt es Weiterentwicklungen der HF-Theorie,
welche die Elektronen-Interaktionen genauer behandeln.
Zu diesen gehören unter anderem die Theorien:
\begin{enumerate}
    \item 
\end{enumerate}

\section{Optimierung}
Trotz der bereits vorhandenen Optimierungen wird beim Berechnen der Resultate deutlich,
dass diese Implementierung noch vergleichsweise ineffizient ist. Vor allem beim Evaluieren
der 2\-/Elektronen\-/Integrale besteht hoher Optimierungsbedarf. Ein konkrete Optimierung wäre z.B.
das Ausnutzen molekularer Symmetrien, über diese lassen sich einzelne Rechnungen mehrfach anwenden
und insgesamt dadurch Laufzeit einsparen.