\documentclass[a4paper, 12pt]{report}

\author{Deniz Güven}

\usepackage[ngerman]{babel}
\usepackage[utf8]{inputenc}
\usepackage[T1]{fontenc}
\usepackage{minted}
\usepackage{tikz}
\usetikzlibrary{arrows, automata}
\usepackage{xcolor}
\usepackage{forest}
%\usepackage[margin=23mm, top=40mm, headheight=24pt, bottom=30mm]{geometry}
\usepackage{graphicx}
\usepackage{fancyhdr}
\usepackage{amsmath, amsfonts, amssymb, amsthm}
\usepackage{mathtools}
\usepackage{stmaryrd}

\begin{document}



\begin{titlepage}
  \begin{center}
    \includegraphics[width=0.5\linewidth]{res/uni.png}
    \vspace{2cm}

    \huge{\textbf{Berechnung von Molekül-Grundzustandsenergien mit Ab-Initio-Methoden}}
    \vspace{1cm}

    \Large
    Fachbereich Informatik

    Bachelorarbeit

    Deniz Güven
    \vfill

    Betreuer: 

    \today
  \end{center}
\end{titlepage}

\tableofcontents

\chapter{Einleitung}
\section{Chemischer Hintergrund}

Atome sind cool\cite{structure_2013}, weil man die essen kann.

(Grundlegende Dinge über Chemie)
Entwicklung der Atom-Modelle bis zur quantenmechanischen Formulierungen
\section{Ziel dieser Bachelor-Arbeit}
\section{Relevanz für die Chemie}
(Reaktionsabläufe besser verstehen ...)



\chapter{Theorie und Methoden}
\section{Allgemeine Theorie}
(die für beide Methoden gilt):
- Schrödinger-Gleichung(Zeitunabhängige)
- Beschreibung/Lösung durch Eigenfunktionen/Eigenwerte
- Verwendete Approximationen/Annahmen (Born-Oppenheimer-Näherung, ...)

\section{Hartree-Fock}
- Herleitung (Variations-Prinzip: Minimierung der Energie,
Schrödingergleichung u Operatoren)
- Lösungsweg über das SCF-Verfahren (Matrix-Darstellung, ...)
- Verwendung von Basisfunktionen (Konstruktion der Wellenfunktion)
- Implementierung (größten Probleme: Integral-Evaluierung und
Matrix-Diagonalisierung)

\section{DFT}
- Herleitung (Nur die Idee/Ergebnisse, da wahrscheinlich über
meinem Niveau)
- Konkrete Umsetzung durch die Kohn-Sham-Gleichung (Terme in der
Schrödinger-Glg. + XC-Funktionale)
- Lösung durch FEM + PINVIT (+ LDA)
- Implemtierung über UG4 LUA

\chapter{Ergebnisse/ Numerische Experimente}
\section{Erklärung der Experimente}
- Eigen-Energien als Benchmark + Moleküle zum Testen (Simple wie
H2O, CH4, ... und Komplexe wie z.b. Benzol, das eine
Elektronen-Delokalisation aufweist)

\section{Experimente(HF, DFT, FULL-CI(exakt) über NWCHEM oder Literatur)}
- Präsentation der Ergebnisse(Graphen, Tabellen, usw.)
- Werden Effekte bei komplexen Molekülen korrekt erfasst?

\section{Vergleich der Methoden/Deutung der Ergebnisse (HF vs. DFT)}
- Genauigkeit, Kosten, Skalierbarkeit, ...

\chapter{Diskussion/Ausblick}
\section{Einordnung von HF und DFT in der Chemie}
-> Andere Klassen von Methoden (zb. semiempirische Methoden)
-> Verbesserung dieser Methoden (Post-Hartree-Fock-Methoden)
-> Eingliederung dieser Methoden in der Praxis (Was kann man mit
diesen Eigenenergien/Funktionen eigentlich machen?).

\section{Wie könnte man von diesem Punkt aus weitermachen?}
-> Code-Optimierung, Anspruchsvoller Methoden implementieren
(aufbauend auf HF), Geometrie-Optimierung, ...

\bibliographystyle{abbrv}
\bibliography{main}

\end{document}