\documentclass[a4paper, 12pt]{report}

\author{Deniz Güven}

\usepackage[ngerman]{babel}
\usepackage[utf8]{inputenc}
\usepackage[T1]{fontenc}
\usepackage{minted}
\usepackage{tikz}
\usetikzlibrary{arrows, automata}
\usepackage{xcolor}
\usepackage{forest}
%\usepackage[margin=23mm, top=40mm, headheight=24pt, bottom=30mm]{geometry}
\usepackage{graphicx}
\usepackage{fancyhdr}
\usepackage{amsmath, amsfonts, amssymb, amsthm}
\usepackage{mathtools}
\usepackage{stmaryrd}

\begin{document}

\begin{titlepage}
  \begin{center}
    \includegraphics[width=0.5\linewidth]{res/uni.png}
    \vspace{2cm}

    \huge{\textbf{Berechnung von Molekül-Grundzustandsenergien 
    mit Ab-Initio-Methoden}}
    \vspace{1cm}

    \Large
    Fachbereich Informatik

    Bachelorarbeit

    Deniz Güven
    \vfill

    Betreuer: 

    \today
  \end{center}
\end{titlepage}

\tableofcontents

%=========================================================================
\chapter{Einleitung}
\section{Chemischer Hintergrund}

TODO
Definition von Chemie

\subsection{Atommodelle}
In der Chemie wurden immer präzisere 
und umfangreichere Beschreibungen des Atoms entwickelt,
zu diesen gehören unter anderen:
TODO
\begin{enumerate}
  \item Demokrit
  \item Dalton
  \item Rutherford
  \item Bohr
  \item Quantenmechanische Modell
\end{enumerate}

\section{Ziel dieser Bachelorarbeit}
Eine wichtige Eigenschaft von Atomen und Molekülen ist die Grundzustandsenergie, 
mit dieser können viele Prozesse in der Chemie,
wie Reaktionsabläufe und Molekülstrukturen, erklärt werden.
Diese Bachelorarbeit befasst sich mit der Berechnung 
dieser Energie von einfachen Atomen und Molekülen innerhalb einer quantenmechanischen Beschreibung.
Dafür wird ein theoretisches Verständis entwickelt 
und diese Theorie dann in einem Programm implementiert.

\section{Relevanz für die Chemie}
TODO Genaue Verwendung von GZs erklären. 

%=========================================================================
\chapter{Theorie und Methoden}
\section{Allgemeine Theorie}

\subsection{Postulate der Quantenmechanik}

TODO cite book chapter

\subsubsection{Wellenfunktionen}
Die sogenannte Wellenfunktion $\Psi(x_1, x_2, \dots, t)$ 
beschreibt den Zustand eines quantenmechanischen Systems vollständig.\cite[S. 20-21]{atkins_friedman_2011}\\\\
Durch ein $\psi$ wird nur der räumliche Anteil dargestellt.

\subsubsection{Operatoren}
Beobachtbare Eigenschaften eines quantenmechanischen Systems 
werden durch sogenannte Operatoren repräsentiert.
Diese Operatoren müssen hermitisch sein und folgende Gleichungen erfüllen:
\begin{flalign*}
  \hat{q}\hat{p}_{q'} - \hat{p}_{q'}\hat{q} &= i\hbar\delta_{qq'}\\
  \hat{q}\hat{q}' - \hat{q}'\hat{q} &= 0 \\
  \hat{p}_{q}\hat{p}_{q'} - \hat{p}_{q'}\hat{p}_{q} &= 0
\end{flalign*}
Wobei $\hat{q}, \hat{q}' \in \{\hat{x}, \hat{y}, \hat{z}\}$ die Orts-Operatoren stellen
und $\hat{p}_q, \hat{p}_{q'}$ die zugehörigen Impuls-Operatoren sind.\cite[S. 21]{atkins_friedman_2011}

\subsubsection{Messungen}
Der Mittelwert von wiederholten Messungen
entspricht dem Erwartungswert des korrespondierenden Operators auf der Wellenfunktion.\\
Der Erwartungswert eines Operators $\hat{o}$ ist gegeben durch:
\begin{equation}
  \left\langle \hat{o} \right\rangle = \frac{\int \psi^* \hat{o} \psi \,d\tau}{\int \psi^* \psi \,d\tau}
  =\frac{\left\langle \psi^* \vert \hat{o} \vert \psi \right\rangle}{\left\langle \psi^* \vert \psi \right\rangle}
\end{equation}\\
Wenn $\psi$ eine Eigenfunktion eines Operators $\hat{o}$ ist, erhält man den Eigenwert als Erwartungswert.
\cite[S. 22-23]{atkins_friedman_2011} 

\subsubsection{Bornsche Wahrscheinlichkeitsinterpretation}
Die Wahrscheinlichkeit ein Teilchen in einem Volumenelement $\,d\tau$ zu finden ist geich
$\vert \psi(x) \vert^2$, wenn $\psi$ normiert ist.\cite[S. 24]{atkins_friedman_2011}

\subsubsection{Die Schrödingergleichung}
Die zeitliche Änderung dieser Wellenfunktion $\Psi(x_1, x_2, \dots, t)$ 
wird durch die Schrödingergleichung beschrieben:
\begin{equation}
  i\hbar\frac{\partial\Psi}{\partial t} = \hat{H}\Psi
\end{equation}\\
Sollte die Zeitabhägigkeit der Wellenfunktion trivial sein (TODO erklären?),
kann die diese in 2 Funktionen zerlegt werden:
\begin{equation*}
  \Psi(x_1, x_2, \dots, t) = \psi(x_1, x_2, \dots) \exp(\frac{-iEt}{\hbar})
\end{equation*}\\
Dabei ist $\psi$ die Lösung für die zeitunabhängige Schrödingergleichung:
\begin{equation}
  \hat{H}\psi = E\psi
\end{equation}
\cite[S. 24-25]{atkins_friedman_2011}
\subsection{Approximationen}
- Verwendete Approximationen/Annahmen (Born-Oppenheimer-Näherung, ...)

\subsection{Variationsformulierung}

\section{Hartree-Fock}
- Herleitung (Variations-Prinzip: Minimierung der Energie,
Schrödingergleichung u Operatoren)

- Lösungsweg über das SCF-Verfahren (Matrix-Darstellung, ...)

- Verwendung von Basisfunktionen (Konstruktion der Wellenfunktion)

- Implementierung (größten Probleme: Integral-Evaluierung und
Matrix-Diagonalisierung)

\section{DFT}
- Herleitung (Nur die Idee/Ergebnisse, da wahrscheinlich über
meinem Niveau)

- Konkrete Umsetzung durch die Kohn-Sham-Gleichung (Terme in der
Schrödinger-Glg. + XC-Funktionale)

- Lösung durch FEM + PINVIT (+ LDA)

- Implemtierung über UG4 LUA

%=========================================================================
\chapter{Ergebnisse/ Numerische Experimente}
\section{Erklärung der Experimente}
- Eigen-Energien als Benchmark + Moleküle zum Testen (Simple wie
H2O, CH4, ... und Komplexe wie z.b. Benzol, das eine
Elektronen-Delokalisation aufweist)

\section{Experimente(HF, DFT, FULL-CI(exakt) über NWCHEM oder Literatur)}
- Präsentation der Ergebnisse(Graphen, Tabellen, usw.)
- Werden Effekte bei komplexen Molekülen korrekt erfasst?

\section{Vergleich der Methoden/Deutung der Ergebnisse (HF vs. DFT)}
- Genauigkeit, Kosten, Skalierbarkeit, ...

%=========================================================================
\chapter{Diskussion/Ausblick}
\section{Einordnung von HF und DFT in der Chemie}
-> Andere Klassen von Methoden (zb. semiempirische Methoden)
-> Verbesserung dieser Methoden (Post-Hartree-Fock-Methoden)
-> Eingliederung dieser Methoden in der Praxis (Was kann man mit
diesen Eigenenergien/Funktionen eigentlich machen?).

\section{Wie könnte man von diesem Punkt aus weitermachen?}
-> Code-Optimierung, Anspruchsvoller Methoden implementieren
(aufbauend auf HF), Geometrie-Optimierung, ...

%=========================================================================
\bibliographystyle{abbrv}
\bibliography{main}

\end{document}